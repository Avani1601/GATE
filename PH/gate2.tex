\documentclass[journal,12pt,twocolumn]{IEEEtran}
\usepackage{cite}
\usepackage{amsmath,amssymb,amsfonts,amsthm}
\usepackage{algorithmic}
\usepackage{graphicx}
\usepackage{textcomp}
\usepackage{xcolor}
\usepackage{listings}
\usepackage{enumitem}
\usepackage{mathtools}
\usepackage{gensymb}
\usepackage{comment}
\usepackage[breaklinks=true]{hyperref}
\usepackage{tkz-euclide}
\usepackage{gvv} 
\usepackage{circuitikz}
\def\inputGnumericTable{} 
\usepackage[latin1]{inputenc} 
\usepackage{color} 
\documentclass{article}
\usepackage{graphicx}
\usepackage{subcaption}

\newtheorem{theorem}{Theorem}[section]
\newtheorem{problem}{Problem}
\newtheorem{proposition}{Proposition}[section]
\newtheorem{lemma}{Lemma}[section]
\newtheorem{corollary}[theorem]{Corollary}
\newtheorem{example}{Example}[section]
\newtheorem{definition}[problem]{Definition}
\newcommand{\BEQA}{\begin{eqnarray}}
\newcommand{\EEQA}{\end{eqnarray}}
\newcommand{\define}{\stackrel{\triangle}{=}}
\theoremstyle{remark}
\newtheorem{rem}{Remark}

\begin{document}

\bibliographystyle{IEEEtran}
\vspace{3cm}

\title{GATE 2022-PH}
\author{EE23BTECH1205 - Avani Chouhan$^{*}$}
\maketitle
\newpage
\bigskip

\renewcommand{\thefigure}{\theenumi}
\renewcommand{\thetable}{\theenumi}

\vspace{3cm}
\textbf{Question : 11} \\
For the Op-Amp circuit shown below, choose the correct output waveform corresponding to the input \( V_{\text{in}} = 1.5 \sin(20 \pi t) \) (in Volts). The saturation voltage for this circuit is \( V_{\text{sat}} = \pm 10 \) V.
\begin{figure}[htb]
\centering
    \input{figs/fig}
    \label{fig:1}
\end{figure}


\begin{enumerate}
  \item[(A)]
  \setcounter{figure}{0}
    \begin{figure}[h]
        \centering
        \includegraphics[width=0.8\linewidth]{figs/1.png}
        \label{fig:ann_label}
    \end{figure}
    
  \item[(B)]
  \setcounter{figure}{1}
    \begin{figure}[h]
        \centering
        \includegraphics[width=0.8\linewidth]{figs/2.png}
        \label{fig:sww_label}
    \end{figure}
    
  \item[(C)]
  \setcounter{figure}{2}
    \begin{figure}[h]
        \centering
        \includegraphics[width=0.8\linewidth]{figs/3.png}
        \label{fig:avv_label}
    \end{figure}
    
  \item[(D)]
  \setcounter{figure}{3}
    \begin{figure}[h]
        \centering
        \includegraphics[width=0.8\linewidth]{figs/4.png}
        \label{fig:abb_label}
    \end{figure}
\end{enumerate}

\hfill{(GATE PH 2022)}\\
\textbf{Solution:} \\
Given circuit is a Schmitt Trigger circuit. In this output will be always saturated, i.e., limited between \( +V_{\text{sat}} \) to \( -V_{\text{sat}} \).
So,answer is (A)
\end{document}


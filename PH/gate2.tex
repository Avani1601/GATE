\documentclass[journal,12pt,twocolumn]{IEEEtran}
\usepackage{cite}
\usepackage{amsmath,amssymb,amsfonts,amsthm}
\usepackage{graphicx}
\usepackage{textcomp}
\usepackage{xcolor}
\usepackage{enumitem}
\usepackage{mathtools}
\usepackage{gensymb}
\usepackage{comment}
\usepackage[breaklinks=true]{hyperref}
\usepackage{tkz-euclide}
\usepackage{circuitikz}

\begin{document}

\bibliographystyle{IEEEtran}
\vspace{3cm}

\title{GATE 2022-PH}
\author{EE23BTECH1205 - Avani Chouhan$^{*}$}
\maketitle
\newpage
\bigskip

\renewcommand{\thefigure}{\theenumi}
\renewcommand{\thetable}{\theenumi}

\vspace{3cm}
\textbf{Question : 11} \\
For the Op-Amp circuit shown below, choose the correct output waveform corresponding to the input \( V_{\text{in}} = 1.5 \sin(20 \pi t) \) (in Volts). The saturation voltage for this circuit is \( V_{\text{sat}} = \pm 10 \) V.
\begin{figure}[htb]
\centering
    \input{figs/fig}
    \label{fig:1}
\end{figure}


\begin{enumerate}
  \item[(A)]
  \setcounter{figure}{0}
    \begin{figure}[h]
        \centering
        \includegraphics[width=0.8\linewidth]{figs/1.png}
        \label{fig:ann_label}
    \end{figure}
    
  \item[(B)]
  \setcounter{figure}{1}
    \begin{figure}[h]
        \centering
        \includegraphics[width=0.8\linewidth]{figs/2.png}
        \label{fig:sww_label}
    \end{figure}
    
  \item[(C)]
  \setcounter{figure}{2}
    \begin{figure}[h]
        \centering
        \includegraphics[width=0.8\linewidth]{figs/3.png}
        \label{fig:avv_label}
    \end{figure}
    
  \item[(D)]
  \setcounter{figure}{3}
    \begin{figure}[h]
        \centering
        \includegraphics[width=0.8\linewidth]{figs/4.png}
        \label{fig:abb_label}
    \end{figure}
\end{enumerate}

\hfill{(GATE PH 2022)}\\
\textbf{Solution:} \\

\begin{table}
  \centering
  
  \begin{tabular}{|c|c|c|}
    \hline
    \textbf{Parameter} & \textbf{Value} \\
    \hline
    $Vin$ & $1.5 \sin(20\pi t)$ \\
    \hline
    $V_{sat}$ & $\pm 10 \, \text{V}$ \\
    \hline
    $V_{out}$ & $10 \times 1.5 \sin(20\pi t)$ \\
    \hline
\end{tabular}


  \caption{Input Parameters}
  \label{tab:PH.11.table1}
\end{table}
\begin{align}
V_{in} &= 1.5 \sin(20\pi t)\\
V_{\text{sat}} &= \pm 10 \, \text{V}\\
I_1 &= I_2\\
\frac{0 - V_{\text{in}}}{2.2 \, \text{k}\Omega} &= \frac{V_{\text{in}} - V_o}{20 \, \text{k}\Omega}\\
\frac{-20}{2.2} &= \frac{V_{\text{in}} - V_o}{V_{\text{in}}}\\
\frac{-20}{2.2} &= 1 - \frac{V_o}{V_{\text{in}}}\\
\frac{V_o}{V_{\text{in}}} &= 1 + \frac{20}{2.2}\\
V_o &\sim 10 V_{\text{in}}\\
V_o &= 10 \times 1.5 \sin(20\pi t)
\end{align}

Output amplitude is greater than $V_{\text{sat}}$, so the voltage saturates at $V_{\text{sat}}$.\\
Therefore, correct answer is (A).
\end{document}

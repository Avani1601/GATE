\documentclass[journal,12pt,twocolumn]{IEEEtran}
\usepackage{cite}
\usepackage{amsmath,amssymb,amsfonts,amsthm}
\usepackage{algorithmic}
\usepackage{graphicx}
\usepackage{textcomp}
\usepackage{xcolor}
\usepackage{listings}
\usepackage{enumitem}
\usepackage{mathtools}
\usepackage{gensymb}
\usepackage{comment}
\usepackage[breaklinks=true]{hyperref}
\usepackage{tkz-euclide}
\usepackage{gvv} 
\def\inputGnumericTable{} 
\usepackage[latin1]{inputenc} 
\usepackage{color} 

\newtheorem{theorem}{Theorem}[section]
\newtheorem{problem}{Problem}
\newtheorem{proposition}{Proposition}[section]
\newtheorem{lemma}{Lemma}[section]
\newtheorem{corollary}[theorem]{Corollary}
\newtheorem{example}{Example}[section]
\newtheorem{definition}[problem]{Definition}
\newcommand{\BEQA}{\begin{eqnarray}}
\newcommand{\EEQA}{\end{eqnarray}}
\newcommand{\define}{\stackrel{\triangle}{=}}
\theoremstyle{remark}
\newtheorem{rem}{Remark}

\begin{document}

\bibliographystyle{IEEEtran}
\vspace{3cm}

\title{GATE 2022-IN}
\author{EE23BTECH1205 - Avani Chouhan$^{*}$}
\maketitle
\newpage
\bigskip

\renewcommand{\thefigure}{\theenumi}
\renewcommand{\thetable}{\theenumi}

\vspace{3cm}
\textbf{Question : 18} \\
A signal \( x(t) \) is band-limited between 100 Hz and 200 Hz. A signal \( y(t) \) is related to \( x(t) \) as follows:\\

\( y(t) = x(2t - 5) \)\\
The statement that is always true is \\

\begin{enumerate}
  \item[(A)] \( y(t) \) is band-limited between 50 Hz and 100 Hz
  \item[(B)] \( y(t) \) is band-limited between 100 Hz and 200 Hz
  \item[(C)] \( y(t) \) is band-limited between 200 Hz and 400 Hz
  \item[(D)] \( y(t) \) is not band-limited 
\end{enumerate}

\hfill{(GATE ST 2022)}\\
\textbf{Solution:} \\

Derivation of Fourier transform of \(x(2t-5)\):
\begin{align}
x(t) &\rightleftharpoons X(\omega) \label{eq1}\\
x(at) &\rightleftharpoons \frac{1}{|a|} X\left(\frac{\omega}{a}\right) \label{eq2}\\
x(2t) &\rightleftharpoons \frac{1}{2} X\left(\frac{\omega}{2}\right) \label{eq3}\\
x(t - t_0) &\rightleftharpoons e^{-j\omega t_0}X(\omega) \label{eq4}\\
x(2t - 5) &\rightleftharpoons e^{-j5\omega} \cdot \frac{1}{2} X\left(\frac{\omega}{2}\right) \label{eq5}
\end{align}

The operation \(x(2t-5)\) compresses time by a factor of 2 and shifts 5 units rightward. This expands the frequency domain, doubling the bandwidth of \(x(t)\) from 100 Hz to 200 Hz to \(y(t)\) between 200 Hz and 400 Hz.\\

Hence, the correct answer is option (C).

\end{document}


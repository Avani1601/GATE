\documentclass[journal,12pt,twocolumn]{IEEEtran}
\usepackage{cite}
\usepackage{amsmath,amssymb,amsfonts,amsthm}
\usepackage{algorithmic}
\usepackage{graphicx}
\usepackage{textcomp}
\usepackage{xcolor}
\usepackage{txfonts}
\usepackage{listings}
\usepackage{enumitem}
\usepackage{mathtools}
\usepackage{gensymb}
\usepackage{comment}
\usepackage[breaklinks=true]{hyperref}
\usepackage{tkz-euclide}
\usepackage{gvv}
\def\inputGnumericTable{}
\usepackage[latin1]{inputenc}
\usepackage{color}
\usepackage{array}
\usepackage{longtable}
\usepackage{calc}
\usepackage{multirow}
\usepackage{hhline}

\newtheorem{theorem}{Theorem}[section]
\newtheorem{problem}{Problem}
\newtheorem{proposition}{Proposition}[section]
\newtheorem{lemma}{Lemma}[section]
\newtheorem{corollary}[theorem]{Corollary}
\newtheorem{example}{Example}[section]
\newtheorem{definition}[problem]{Definition}
\newcommand{\BEQA}{\begin{eqnarray}}
\newcommand{\EEQA}{\end{eqnarray}}
\newcommand{\define}{\stackrel{\triangle}{=}}
\theoremstyle{remark}
\newtheorem{rem}{Remark}

\begin{document}

\bibliographystyle{IEEEtran}
\vspace{3cm}

\title{GATE 2023-EC}
\author{EE23BTECH1205 - Avani Chouhan$^{*}$% <-this % stops a space
}
\maketitle
\newpage
\bigskip

\renewcommand{\thefigure}{\theenumi}
\renewcommand{\thetable}{\theenumi}

\vspace{3cm}
\textbf{Question : 14} \\
The value of the contour integral, $\oint_C \frac{z + 2}{z^2 + 2z + 2} \, dz$, where the contour $C$ is $\{ z : |z + 1 - \frac{3}{2}i| = 1 \}$, taken in the counter clockwise direction, is \\

\begin{enumerate}
  \item[(A)] $-\pi(1+j) $
  \item[(B)] $\pi(1+j)$
  \item[(C)] $\pi(1-j) $
  \item[(D)] $-\pi(1-j)$
\end{enumerate}

\hfill{(GATE ST 2023)}\\
\textbf{Solution:}
To evaluate the contour integral using Cauchy's residue theorem, first find the residues of the function inside the contour and then sum them up.

The function inside the contour is $f(z) = \frac{z + 2}{z^2 + 2z + 2}$. We can factorize the denominator to find its roots:

\begin{align}
z^2 + 2z + 2 &= (z+1+i)(z+1-i)
\end{align}

Thus, the roots are $-1+i$ and $-1-i$.

Now, find the residues at these poles. The residue of a function $f(z)$ at a simple pole $z_0$ is given by:

\begin{align}
\text{Res}(f(z), z_0) &= \lim_{z \to z_0} (z - z_0)f(z)
\end{align}

Let's first find the residue at $z = -1 + i$:

\begin{align}
\text{Res}(f(z), -1+i) &= \lim_{z \to -1+i} (z - (-1+i))\frac{z + 2}{z^2 + 2z + 2} \\
&= \lim_{z \to -1+i} \frac{z + 2}{(z+1-i)(z+1+i)} \\
&= \frac{-1+i + 2}{(-1+i+1-i)(-1+i+1+i)} \\
&= \frac{1+i}{2i} = \frac{1}{2}(1+i)
\end{align}

Similarly, the residue at $z = -1 - i$ can be found as:

\begin{align}
\text{Res}(f(z), -1-i) &= \lim_{z \to -1-i} (z - (-1-i))\frac{z + 2}{z^2 + 2z + 2} \\
&= \frac{1-i}{2i} = \frac{1}{2}(1-i)
\end{align}

From the equation \eqref{C.1.9} and\eqref{C.1.10}\\
Now, by Cauchy's residue theorem, the contour integral is equal to $2\pi i$ times the sum of residues inside the contour.

\begin{align}
\oint_C \frac{z + 2}{z^2 + 2z + 2} \, dz &= 2\pi i \left(\text{Res}(f(z), -1+i) + \text{Res}(f(z), -1-i)\right) \\
&= 2\pi i \left(\frac{1}{2}(1+i) + \frac{1}{2}(1-i)\right) \\
&= 2\pi i \cdot \frac{1}{2}(1+i+1-i) \\
&= 2\pi i \cdot \frac{1}{2}(2) \\
&= \pi i (2) \\
&= \pi (2i) \\
&= \pi (0 + 2i) \\
&= \pi (1 + i)
\end{align}

So, the correct answer is (B) $\pi(1 + i)$.

\end{document}

